%\documentclass{beamer}
\documentclass[handout]{beamer}

\usepackage{booktabs}
\usepackage{siunitx}
\usepackage{circuitikz}
\usepackage{tikz}
\usepackage{pgfplots}
\usepackage{pgfplotstable}
\usepackage{ifthen}
\usepackage[absolute,overlay]{textpos}
\usepackage{todonotes}
\usepackage{listings}
\usepackage{graphicx}
\usepackage{subfig}
\usepackage{xcolor}
\usepackage[export]{adjustbox}
\usepackage{algorithm}
\usepackage{algorithmicx}
\usepackage{caption}
\usepackage[noend]{algpseudocode}

\usepackage{minted} % source-code highlighting

\captionsetup{justification=centering}

\usetikzlibrary{matrix,backgrounds,shapes,shapes.callouts,arrows,positioning,calc,snakes,fit}
\usetikzlibrary{circuits.logic.US}
\usepgflibrary{decorations.markings}

\AtBeginSection[]{
  \begin{frame}
  \vfill
  \centering
  \begin{beamercolorbox}[sep=8pt,center,shadow=true,rounded=true]{title}
    \usebeamerfont{title}\insertsectionhead\par%
  \end{beamercolorbox}
  \vfill
  \end{frame}
}

\newcommand*{\DOT}{.}
\DeclareGraphicsExtensions{.pdf,.jpeg,.png,.eps}

\setbeamertemplate{sidebar right}{}
\setbeamertemplate{footline}{%
  \hfill\usebeamertemplate***{navigation symbols}
\hspace{1cm}\insertframenumber{}/\inserttotalframenumber}

\makeatletter
\let\slideno\beamer@slideinframe
\makeatother


\begin{document}

\pgfdeclarelayer{bg}    % declare background layer
\pgfsetlayers{bg,main}  % set the order of the layers (main is the standard layer)

\thispagestyle{empty}
\begin{frame}
  %\frametitle{Lab3 $\rightarrow$ Systolic Core}
  \begin{center}
  \scalebox{0.9}{
  \begin{tikzpicture}
    \pgfmathsetmacro{\Nx}{3}
    \pgfmathsetmacro{\Ny}{3}
    \foreach \x in {0,...,\Nx} {
      \foreach \y in {0,...,\Ny} {
    	\pgfmathtruncatemacro{\xx}{\Nx-\x}
    	\pgfmathtruncatemacro{\yy}{\Ny-\y}
        \node [draw,fill=blue!20,minimum height=0.75cm,minimum
          width=0.75cm,align=center] (mac\x\y) at
	  (1.2*\x,1.2*\y) {\tiny pe.v\\[-1ex]\tiny pe[\yy][\x]};
      }
    }
    \pgfmathtruncatemacro{\Nym}{\Ny-1}
    \foreach \x in {0,...,\Nx} {
      \foreach \y in {0,...,\Nym} {
        \pgfmathtruncatemacro{\yp}{\y+1}
        \draw [<-] (mac\x\y.north) -- (mac\x\yp.south);
      }
    }
    \pgfmathtruncatemacro{\Nxm}{\Nx-1}
    \foreach \x in {0,...,\Nxm} {
      \foreach \y in {0,...,\Ny} {
	\pgfmathtruncatemacro{\xp}{\x+1}
        \draw [->] (mac\x\y.east) -- (mac\xp\y.west);
      }
    }
    \foreach \y in {0,...,\Ny} {
      \pgfmathtruncatemacro{\yy}{\Ny-\y}
      \node [fill=red!20,minimum height=0.5cm,minimum
        width=0.5cm] (rama\y) at
	(-1.2,1.2*\y) {\tiny A[\yy]};
      %\draw [->] (mac\Nx\y.east) -- ++(1,0);
      \draw [<-] (mac0\y.west) -- (rama\y.east);
    }
    \foreach \x in {0,...,\Nx} {
      \node [fill=red!20,minimum height=0.5cm,minimum
        width=0.5cm] (ramb\x) at
	(1.2*\x,1.2*\Ny+1) {\tiny B[\x]};
      \draw [<-] (mac\x\Ny.north) -- (ramb\x);
    }
    \node [draw,fill=blue!20,minimum height=0.25cm,minimum
      width=0.5cm,align=center] (ctrlmm) at ($(mac00)!0.5!(mac10)+(0,-0.75)$) {\tiny
      counter.v};
    \begin{pgfonlayer}{bg}
      \begin{scope}
	\node (systolic) [draw=black,fill=blue!05,fit=(ctrlmm) (mac00)
		(mac\Nx\Ny),rounded corners=0.25cm,inner
		sep=5pt,label={[xshift=5ex,yshift=3ex]below:\tiny systolic.sv}] {};
      \end{scope}
    \end{pgfonlayer}
  \end{tikzpicture}
  }
  \end{center}
\end{frame}
\end{document}

\begin{frame}
  \frametitle{Course Goal $\rightarrow$ Design+Implement systolic array on FPGA}
  \begin{center}
  \scalebox{0.9}{
  \begin{tikzpicture}
    \pgfmathsetmacro{\Nx}{3}
    \pgfmathsetmacro{\Ny}{3}
    \foreach \x in {0,...,\Nx} {
      \foreach \y in {0,...,\Ny} {
        \node [draw,fill=blue!20,minimum height=0.75cm,minimum
          width=0.75cm] (mac\x\y) at
	  (1.5*\x,1.5*\y) {\tiny MAC};
      }
    }
    \pgfmathtruncatemacro{\Nym}{\Ny-1}
    \foreach \x in {0,...,\Nx} {
      \foreach \y in {0,...,\Nym} {
        \pgfmathtruncatemacro{\yp}{\y+1}
        \draw [<-] (mac\x\y.north) -- (mac\x\yp.south);
      }
    }
    \pgfmathtruncatemacro{\Nxm}{\Nx-1}
    \foreach \x in {0,...,\Nxm} {
      \foreach \y in {0,...,\Ny} {
	\pgfmathtruncatemacro{\xp}{\x+1}
        \draw [->] (mac\x\y.east) -- (mac\xp\y.west);
      }
    }
    \foreach \y in {0,...,\Ny} {
      \node [draw,fill=red!20,minimum height=0.75cm,minimum
        width=0.5cm] (ramd\y) at
	(1.5*\Nx+1.5,1.5*\y) {\tiny RAM};
      \node [draw,fill=red!20,minimum height=0.75cm,minimum
        width=0.5cm] (rama\y) at
	(-1.5,1.5*\y) {\tiny RAM};
      \draw [->] (mac\Nx\y.east) -- (ramd\y);
      \draw [<-] (mac0\y.west) -- (rama\y);
    }
    \foreach \x in {0,...,\Nx} {
      \node [draw,fill=red!20,minimum height=0.75cm,minimum
        width=0.5cm] (ramb\x) at
	(1.5*\x,1.5*\Ny+1.5) {\tiny RAM};
      \draw [<-] (mac\x\Ny.north) -- (ramb\x);
    }
    % FSMs
    \node [draw,fill=yellow!20,minimum height=0.25cm,minimum
      width=0.5cm] (ctrlab) at (-1.5,-1.0) {\tiny Ctrl};
    \node [draw,fill=yellow!20,minimum height=0.25cm,minimum
      width=0.5cm] (ctrld) at (1.5*\Nx+1.5,-1.0) {\tiny Ctrl};
    \node [draw,fill=yellow!20,minimum height=0.25cm,minimum
      width=0.5cm] (ctrlmm) at ($(mac00)!0.5!(mac\Nx0)+(0,-1.0)$) {\tiny Ctrl};


    \begin{pgfonlayer}{bg}
      \begin{scope}
	\node (mm) [draw=black,fill=green!05,fit=(rama0) (ramb\Nx) (ramd0) (ctrlab) (ctrld),rounded corners=0.25cm,inner sep=15pt] {};
	\node (ramb) [draw=black,fill=red!05,fit=(ramb0) (ramb\Nx),rounded corners=0.25cm,inner sep=5pt] {};
	\node (rama) [draw=black,fill=red!05,fit=(ctrlab) (rama0) (rama\Ny),rounded corners=0.25cm,inner sep=5pt] {};
	\node (ramd) [draw=black,fill=red!05,fit=(ctrld) (ramd0) (ramd\Ny),rounded corners=0.25cm,inner sep=5pt] {};
	\node (systolic) [draw=black,fill=blue!05,fit=(ctrlmm) (mac00) (mac\Nx\Ny),rounded corners=0.25cm,inner sep=5pt] {};
      \end{scope}
    \end{pgfonlayer}

    \draw [<-,ultra thick] (rama0) -- ++(-1.25,0);
    \draw [->,ultra thick] (ramd0) -- ++(1.25,0);
    \draw [->,rounded corners=0.25cm] (rama) |- (ramb);

%    https://tex.stackexchange.com/questions/181098/semitransparent-outside-clipped-region
    \only<2|handout:2>{
      \fill[white,fill opacity=.75]
	(current bounding box.south west) rectangle (current bounding box.north east)
      (mac00) circle[radius=0.5cm+.5\pgflinewidth];
      \node[draw,fill=white,ellipse callout,callout absolute
	  pointer={(mac00.center)},align=center] at (2,2) {Lab1,\\ Lab Manual:
	Simulation\\ Lecture: Simulation};
    }
    \only<3|handout:3>{
      \fill[white,fill opacity=.75]
	(current bounding box.south west) rectangle (current bounding box.north east)
	(ctrlmm) circle[radius=0.5cm+.5\pgflinewidth];
      \node[draw,fill=white,ellipse callout,callout absolute
	  pointer={(ctrlmm.center)},align=center] at (2,2) {Lab2,\\ Lab Manual:
	Implementation\\ Lecture: Simulation, \\ Synthesis};
    }

    \only<4|handout:4>{
      \fill[white,fill opacity=.75]
	(current bounding box.south west) rectangle (current bounding box.north east)
	(systolic) circle[radius=3.5cm+.5\pgflinewidth];
      \node[draw,fill=white,ellipse callout,callout absolute
	  pointer={(systolic.center)},align=center] at (4,5) {Lab3,\\ Lecture:
	Pipelining, \\ Scheduling\\ Top-Down Synthesis};
    }

    \only<5|handout:5>{
%      \fill[white,fill opacity=.75]
%	(current bounding box.south west) rectangle (current bounding box.north east)
%	(mm) circle[radius=4.5cm+.5\pgflinewidth];
      \node[draw,fill=white,ellipse callout,callout absolute
	  pointer={(systolic.center)},align=center] at (3,5) {Lab4,\\ Lab
	  Manual: Pynq FPGA, \\ Floorplanning\\ Lecture:
	FPGA Arch., \\ Memories, Timing };
    }


  \end{tikzpicture}
  }
  \end{center}
\end{frame}

\end{document}
